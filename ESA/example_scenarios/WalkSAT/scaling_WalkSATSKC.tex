%% template-AutoScaling.tex
%% Author: Zongxu Mu, Yasha Pushak
%% This is the LaTeX template file for Empirical Scaling Analyser (ESA) v1.1.
%% ESA takes the template, replace variables with their corresponding values,
%%	and generates an output file named AutoScaling.tex.
%% You may compile this file alone without running ESA to see how the output looks like.
%% Variables are surrounded by ``@@''s
%% Supported variable names include:
%%	- algName, e.g., ``WalkSAT/SKC''
%%	- instName, e.g., ``random 3SAT at phase transition''
%%	- models, e.g., ``\\begin{itemize}\n\\item $Exp\\left[a,b\\right]\\left(n\\right)=a\\cdot b^{n}$ \\quad{}(2-parameter exponential);\n\\end{itemize}''
%%	- numBootstrapSamples, the number of bootstrap samples used in the analysis, e.g., 1000
%%	- numSizes, the number of sizes used in the analysis, e.g., 12
%%	- largestSupportSize, e.g., 500
%%	- table-Details-dataset, e.g., ``\\input{table_Details-dataset}''
%%	- table-Details-dataset, e.g., ``\\input{table_Details-dataset}''
%%	- table-Fitted-models, e.g., ``\\begin{tabular}{ccccc} 
\hline 
 &  & \multirow{2}{*}{Model} & RMSE  & RMSE\tabularnewline 
 &  &  & (support)  & (challenge)\tabularnewline 
\hline 
\hline 
\multirow{3}{*}{p15S363} & EXP. Model & $0.95094\times 1.0017^{x}$ & $0.34618$ & $30.319$ \tabularnewline 
 & Poly. Model & $4.3589\times10^{-8}\times x^{2.675}$ & $0.84016$ & $34.759$ \tabularnewline 
 & SQRTEXP. Model & $\mathbf{0.06958\times 1.145^{\sqrt{x}}}$ & $\mathbf{0.39499}$ & $\mathbf{19.033}$ \tabularnewline 
\hline 
\end{tabular} 

''
%%	- figure-fittedModels, e.g., ``\\includegraphics[width=0.8\textwidth]{fittedModels_loglog}''
%%	- supportSizes, the sizes used for fitting the models, e.g., ``200, 250, 300, 350, 400, 450, 500''
%%	- challengeSizes, e.g., ``600, 700, 800, 900, 1000''
%%	- table-Bootstrap-intervals-of-parameters, e.g., ``\\begin{tabular}{cc|cc} 
\hline 
Solver  & Model  & Confidence interval of $a$  & Confidence interval of $b$ \tabularnewline 
\hline 
\multirow{3}{*}{WalkSAT/SKC} & Exp. & $\left[0.0004476,0.0010113\right]$ & $\left[1.007,1.0091\right]$ \tabularnewline 
 & RootExp. & $\left[1.0905\times10^{-5},6.1897\times10^{-5}\right]$ & $\left[1.3243,1.4433\right]$ \tabularnewline 
 & Poly. & $\left[4.5554\times10^{-12},1.011\times10^{-9}\right]$ & $\left[2.7812,3.6834\right]$ \tabularnewline 
\hline 
\end{tabular} 

''
%%	- table-Bootstrap-intervals, e.g., ``\\input{table_Bootstrap-intervals}''
%%	- analysisSummary, e.g., ``observed median running times are consistent with the polynomial scaling model''
\documentclass[british]{article}
\usepackage[T1]{fontenc}
\usepackage[latin9]{inputenc}
\usepackage{geometry}
\geometry{verbose,tmargin=3.5cm,bmargin=3.5cm,lmargin=3cm,rmargin=3cm}
\usepackage{array}
\usepackage{multirow}
\usepackage{amstext}
\usepackage{graphicx}
\usepackage[usenames, dvipsnames]{color}

\newcommand{\updatedYP}[1]{\textcolor{Purple}{#1}}
\newcommand{\yp}[1]{\textcolor{red}{#1}}
\newcommand{\orange}[1]{\textcolor{BurntOrange}{#1}}
\newcommand{\evalModels}[1]{\textcolor{green}{#1}}
\newcommand{\bestBoot}[1]{\textcolor{blue}{#1}}

\newcommand{\medianInterval}[1]{}
\newcommand{\randomizedAlgorithm}[1]{}


\makeatletter

%%%%%%%%%%%%%%%%%%%%%%%%%%%%%% LyX specific LaTeX commands.
%% Because html converters don't know tabularnewline
\providecommand{\tabularnewline}{\\}

%%%%%%%%%%%%%%%%%%%%%%%%%%%%%% User specified LaTeX commands.

\title{On the empirical scaling of running time of WalkSAT/SKC for solving random 3-SAT instances at the phase transition}
\author{Empirical Scaling Analyser}

\makeatother

\usepackage{babel}
\begin{document}
\maketitle %


\section{Introduction}

This is the automatically generated report on the empirical scaling
of the running time of WalkSAT/SKC for solving random 3-SAT instances at the phase transition.


\section{Methodology}

\label{sec:Methodology}

% models, model fitting
For our scaling analysis, we considered the following parametric models:
\begin{itemize} 
\item $Exp\left[a,b\right]\left(n\right)=a\times b^{x}$ \quad{}(2-parameter Exp)\item $RootExp\left[a,b\right]\left(n\right)=a\times b^{\sqrt{x}}$ \quad{}(2-parameter RootExp)\item $Poly\left[a,b\right]\left(n\right)=a\times x^{b}$ \quad{}(2-parameter Poly)\end{itemize}
% \begin{itemize}
% \item $Exp\left[a,b\right]\left(n\right)=a\cdot b^{n}$ \quad{}(2-parameter exponential);
% \item $RootExp\left[a,b\right]\left(n\right)=a\cdot b^{\sqrt{n}}$ \quad{}(2-parameter root-exponential);
% \item $Poly\left[a,b\right]\left(n\right)=a\cdot n^{b}$ \quad{}(2-parameter polynomial).
% \end{itemize}
Note that the approach could be easily extended to other scaling models.
For fitting parametric scaling models to observed data, we used the
non-linear least-squares Levenberg-Marquardt algorithm.

% what we fitted the models to, how we assessed model fit
Models were fitted to performance observations in the form of medians
of the distributions of running times over sets of instances for given
$n$, the instance size.
\randomizedAlgorithm{
Since WalkSAT/SKC is a randomized algorithm, we analyzed the per-instance median of 1 independent runs for each instance. This means that models were fitted to medians of per-instance medians. Similarly, the running time statistics reported throughout this report are statistics of per-instance medians.
}
%Compared to the mean, the median has two
%advantages: it is statistically more stable and immune to the presence
%of a certain amount of timed-out runs.
To assess the fit of a given
scaling model to observed data, we used root-mean-square error (RMSE).

\medianInterval{
Due to the instances for which the running times are unknown,
there is uncertainty about the
precise location of the medians of the running time distributions
at each such $n$, and we can only provide bounds on
those medians instead. Closely following \cite{dubois2015on}, we calculate
these bounds based on the best-and worst-case scenarios, in which all instances with unknown running times
are easiest or hardest, respectively.
We note that these are not confidence
intervals, since we can guarantee the actual median running
times to lie within them.
We also calculate RMSEs and confidence intervals based on these bounds.
}

% bootstrap confidence intervals
Closely following \cite{hoos2009bootstrap,hoos2014empirical}, we
computed 95\% bootstrap confidence intervals for the performance predictions
obtained from our scaling models, based on 100 bootstrap samples
per instance set and 100 automatically fitted variants of each scaling
model.
\bestBoot{To extend this idea, we calculated support and challenge RMSEs for each of the fitted models' predictions and the corresponding
bootstrap samples of the observed data. We used these bootstrap sample
RMSEs to calculate median and 95\% confidence intervals of the support and challenge RMSEs for each model.
\medianInterval{In order to handle the unknown running times, we used the geometric means of the median intervals for each instance size to calculate the median RMSEs.
However, to better capture both sources of uncertainty for the bootstrap intervals, we calculated the RMSE interval upper and lower bounds by computing 97.5 and 2.5 quantiles
for the set of upper bounds and the set of lower bounds on the RMSEs, respectively.}}
\randomizedAlgorithm{Since this analysis was performed on per-instance medians, we also computed these statistics on nested, per-instance bootstrap samples, by first computing medians for 10 bootstrap samples for each instance and then randomly selecting one of the per-instance medians when needed.}


\updatedYP{
In the following, we say that a scaling model prediction is in-consistent
with observed data if the bootstrap confidence interval for the observed data
is disjoint from the bootstrap confidence interval for the
predicted median \randomizedAlgorithm{of per-instance
median}
running times; we say that a scaling model prediction is weakly consistent
with the observed data if the bootstrap confidence interval for the prediction overlaps with the bootstrap confidence interval for the observed data;
%\medianInterval{we say that a scaling model prediction is consistent
%with observed data, if the interval for observed median \randomizedAlgorithm{of per-instance median} running times
%overlaps with the
%bootstrap confidence interval for predicted running times;}
and, we say that a scaling model is strongly consistent with observed
data, if the bootstrap confidence interval for the observed median \randomizedAlgorithm{of per-instance medians} is fully contained
within the bootstrap confidence interval for predicted
running times.}
Also, we define the residue of a model at a given size as the observed
point estimate less the predicated value.


\section{Dataset Description}

The dataset contains running times of the WalkSAT/SKC algorithm solving
12 sets of instances of different sizes \randomizedAlgorithm{with 1 independent runs per instance}. We split the running times into
two categories, support ($n\leq500$) and challenge ($n>500$). The
details of the dataset can be found in Tables \ref{tab:Details-dataset-support}
and \ref{tab:Details-dataset-challenge}.
\begin{table*}
\noindent \begin{centering}
\begin{tabular}{c|cccc} 
\hline 
$n$ & 200 & 250 & 300 & 350 \tabularnewline 
\hline 
\# instances & 601 & 589 & 633 & 558 \tabularnewline 
\# running times & 601 & 589 & 633 & 558 \tabularnewline 
mean & $0.006527$ & $0.01671$ & $0.04785$ & $0.07433$ \tabularnewline 
coefficient of variation & $1.932$ & $2.708$ & $7.148$ & $4.636$ \tabularnewline 
Q(0.1) & $0.000584$ & $0.001108$ & $0.001648$ & $0.002231$ \tabularnewline 
Q(0.25) & $0.000994$ & $0.001882$ & $0.003173$ & $0.004306$ \tabularnewline 
median & $0.002093$ & $0.004457$ & $0.007494$ & $0.01094$ \tabularnewline 
Q(0.75) & $0.005678$ & $0.01213$ & $0.02102$ & $0.02984$ \tabularnewline 
Q(0.9) & $0.01572$ & $0.03676$ & $0.05997$ & $0.0899$ \tabularnewline 
\hline 
\end{tabular} 
\medskip{} 

\begin{tabular}{c|ccc} 
\hline 
$n$ & 400 & 450 & 500 \tabularnewline 
\hline 
\# instances & 579 & 572 & 578 \tabularnewline 
\# running times & 579 & 572 & 578 \tabularnewline 
mean & $0.2162$ & $0.2634$ & $2.171$ \tabularnewline 
coefficient of variation & $8.165$ & $6.233$ & $17.97$ \tabularnewline 
Q(0.1) & $0.003448$ & $0.005009$ & $0.006445$ \tabularnewline 
Q(0.25) & $0.007598$ & $0.01004$ & $0.01438$ \tabularnewline 
median & $0.01825$ & $0.02414$ & $0.03651$ \tabularnewline 
Q(0.75) & $0.05361$ & $0.08692$ & $0.1295$ \tabularnewline 
Q(0.9) & $0.2451$ & $0.3535$ & $0.4501$ \tabularnewline 
\hline 
\end{tabular} 
\medskip{} 


\par\end{centering}

\caption{\label{tab:Details-dataset-support} Details of the running time dataset used as support data for model fitting. \randomizedAlgorithm{The reported statistics are of the per-instance median running times.}}
\end{table*}

\begin{table*}
\noindent \begin{centering}
\begin{tabular}{c|cc} 
\hline 
$n$ & 2500 & 3000 \tabularnewline 
\hline 
\# instances & 100 & 100 \tabularnewline 
\# running times & 100 & 100 \tabularnewline 
mean & $\infty $ & $\infty $ \tabularnewline 
coefficient of variation & $N/A$ & $N/A$ \tabularnewline 
Q(0.1) & $28.7$ & $46.02$ \tabularnewline 
Q(0.25) & $32.01$ & $70.67$ \tabularnewline 
median & $41.9$ & $135$ \tabularnewline 
Q(0.75) & $76.26$ & $249.2$ \tabularnewline 
Q(0.9) & $155.7$ & $\infty $ \tabularnewline 
\hline 
\end{tabular} 
\medskip{} 


\par\end{centering}

\caption{\label{tab:Details-dataset-challenge} Details of the running time dataset used as challenge data for model fitting. \randomizedAlgorithm{The reported statistics are of the per-instance median running times.}}
\end{table*}

%
% Figure \ref{fig:CDFs} shows the distributions of the running times of
% WalkSAT/SKC solving random 3-SAT instances at the phase transition.
% \begin{figure*}[tb]
% \begin{centering}
% \includegraphics[width=0.8\textwidth]{cdfs}
% % \includegraphics[width=0.8\textwidth]{cdfs}
% \par\end{centering}
%
% \noindent \centering{}\caption{\label{fig:CDFs} Distribution of running times across instance sets for
% WalkSAT/SKC.}
% \end{figure*}
%
%

\section{Empirical Scaling of Solver Performance}

\label{sec:Results}

We first fitted our parametric scaling models to the medians of the \randomizedAlgorithm{per-instance median} running times
of WalkSAT/SKC, as described in Section \ref{sec:Methodology}. The
models were fitted using the medians of the \randomizedAlgorithm{per-instance median} running times for $200\leq n\leq 500$
(support) and later challenged with the medians of the \randomizedAlgorithm{per-instance median} running times for $600\leq n\leq 1000$.
This resulted in the models, shown along with RMSEs on support and
challenge data, shown in Table~\ref{tab:Fitted-models}.
\begin{table}[tb]
\begin{centering}
\begin{tabular}{ccccc} 
\hline 
 &  & \multirow{2}{*}{Model} & RMSE  & RMSE\tabularnewline 
 &  &  & (support)  & (challenge)\tabularnewline 
\hline 
\hline 
\multirow{3}{*}{p15S363} & EXP. Model & $0.95094\times 1.0017^{x}$ & $0.34618$ & $30.319$ \tabularnewline 
 & Poly. Model & $4.3589\times10^{-8}\times x^{2.675}$ & $0.84016$ & $34.759$ \tabularnewline 
 & SQRTEXP. Model & $\mathbf{0.06958\times 1.145^{\sqrt{x}}}$ & $\mathbf{0.39499}$ & $\mathbf{19.033}$ \tabularnewline 
\hline 
\end{tabular} 


% \begin{tabular}{ccccc} 
\hline 
 &  & \multirow{2}{*}{Model} & RMSE  & RMSE\tabularnewline 
 &  &  & (support)  & (challenge)\tabularnewline 
\hline 
\hline 
\multirow{3}{*}{p15S363} & EXP. Model & $0.95094\times 1.0017^{x}$ & $0.34618$ & $30.319$ \tabularnewline 
 & Poly. Model & $4.3589\times10^{-8}\times x^{2.675}$ & $0.84016$ & $34.759$ \tabularnewline 
 & SQRTEXP. Model & $\mathbf{0.06958\times 1.145^{\sqrt{x}}}$ & $\mathbf{0.39499}$ & $\mathbf{19.033}$ \tabularnewline 
\hline 
\end{tabular} 


\par\end{centering}

\caption{\label{tab:Fitted-models}Fitted models of the medians of the \randomizedAlgorithm{per-instance median} running times and RMSE
values (in CPU sec). The models yielding the most
accurate predictions (as per RMSEs on challenge data) are shown in
boldface.}
\end{table}
In addition, we illustrate the fitted models of WalkSAT/SKC in Figure~\ref{fig:Fitted-models},
and the residues for the models in Figure~\ref{fig:Fitted-residues}.
\begin{figure}[tb]
\noindent \begin{centering}
\includegraphics[width=0.8\textwidth]{fittedModels}
% \includegraphics[width=0.8\textwidth]{fittedModels}
\par\end{centering}

\caption{\label{fig:Fitted-models} Fitted models of the medians of the \randomizedAlgorithm{per-instance median} running times.
The models are fitted with the medians of the \randomizedAlgorithm{per-instance median} running times of
WalkSAT/SKC solving the set of random 3-SAT instances at the phase transition
of $200\leq n\leq 500$ variables, and are challenged by the medians of the \randomizedAlgorithm{per-instance median}
running times of $600\leq n\leq 1000$ variables.}
\end{figure}


\begin{figure}[tb]
\noindent \begin{centering}
\includegraphics[width=0.8\textwidth]{fittedResidues}
% \includegraphics[width=0.8\textwidth]{fittedResidues}
\par\end{centering}

\caption{\label{fig:Fitted-residues} Residues of the fitted models of the medians of the \randomizedAlgorithm{per-instance median} running times. }
\end{figure}


But how much confidence should we have in these models? Are the RMSEs
small enough that we should accept them? To answer this question,
we assessed the fitted models using the bootstrap approach outlined
in Section~\ref{sec:Methodology}. Table~\ref{tab:Bootstrap-intervals-of-parameters}
shows the bootstrap intervals of the model parameters,
\bestBoot{Table~\ref{tab:Bootstrap-model-RMSE}
shows the bootstrap intervals of the model prediction RMSEs},
and Table~\ref{tab:Bootstrap-intervals-support}
contains the bootstrap intervals for the support data.
Challenging the models with extrapolation, as shown in
\evalModels{Table~\ref{tab:Bootstrap-intervals-challenge}, it is concluded that
the Exp model over-estimates the data, the RootExp model tends to fit the data, and the Poly model fits the data very well
(as also illustrated in Figure~\ref{fig:Fitted-models}).
We base these statements on an analysis of the fraction of predicted bootstrap intervals that are strongly consistent, weakly consistent and disjoint from the observed bootstrap intervals for the challenge data. To provide stronger emphasis for the largest instance sizes, we also consider these fractions for the largest half of the challenge instance sizes. To be precise, we say a model tends to fit the data if 90\% or more of the predicted bootstrap intervals (or the larger half of the predicted intervals) are weakly consistent with the observed data; we say a model over-estimates the data if more than 70\% of the predicted bootstrap intervals (or more than 70\% of the larger half of the predicted bootstrap intervals) are disjoint from the observed bootstrap intervals and are above the observed intervals; and we say a model fits the data very well if 90\% or more of the predicted bootstrap intervals (or the larger half of the predicted intervals) are strongly consistent with the observed data and at least 90\% of the predicted bootstrap intervals are weakly consistent with the observed data. }
\begin{table*}[tb]
\noindent \begin{centering}
\begin{tabular}{cc|cc} 
\hline 
Solver  & Model  & Confidence interval of $a$  & Confidence interval of $b$ \tabularnewline 
\hline 
\multirow{3}{*}{WalkSAT/SKC} & Exp. & $\left[0.0004476,0.0010113\right]$ & $\left[1.007,1.0091\right]$ \tabularnewline 
 & RootExp. & $\left[1.0905\times10^{-5},6.1897\times10^{-5}\right]$ & $\left[1.3243,1.4433\right]$ \tabularnewline 
 & Poly. & $\left[4.5554\times10^{-12},1.011\times10^{-9}\right]$ & $\left[2.7812,3.6834\right]$ \tabularnewline 
\hline 
\end{tabular} 


% \begin{tabular}{cc|cc} 
\hline 
Solver  & Model  & Confidence interval of $a$  & Confidence interval of $b$ \tabularnewline 
\hline 
\multirow{3}{*}{WalkSAT/SKC} & Exp. & $\left[0.0004476,0.0010113\right]$ & $\left[1.007,1.0091\right]$ \tabularnewline 
 & RootExp. & $\left[1.0905\times10^{-5},6.1897\times10^{-5}\right]$ & $\left[1.3243,1.4433\right]$ \tabularnewline 
 & Poly. & $\left[4.5554\times10^{-12},1.011\times10^{-9}\right]$ & $\left[2.7812,3.6834\right]$ \tabularnewline 
\hline 
\end{tabular} 



\par\end{centering}
\caption{\label{tab:Bootstrap-intervals-of-parameters} 95\% bootstrap intervals
of model parameters for the medians of the \randomizedAlgorithm{per-instance median} running times}

%Group tables 4 and 5 together.
%\end{table*}
%\begin{table*}[tb]
\bigskip

\noindent \begin{centering}
\begin{tabular}{cc|cc|cc} 
\hline 
 \multirow{2}{*}{Solver} & \multirow{2}{*}{Model} & \multicolumn{2}{c|}{Support RMSE}  & \multicolumn{2}{c}{Challenge RMSE} \tabularnewline & & Median & Confidence Interval & Median & Confidence Interval \tabularnewline\hline 
\hline 
\multirow{3}{*}{WalkSAT/SKC} & Exp. & $0.0013414$ & $\left[0.00066148,0.0023947\right]$ & $0.76758$ & $\left[0.34781,1.633\right]$ \tabularnewline 
 & RootExp. & $0.0010909$ & $\left[0.00044449,0.0025122\right]$ & $0.16244$ & $\left[0.026794,0.4174\right]$ \tabularnewline 
 & Poly. & $\mathbf{0.001276}$ & $\mathbf{\left[0.00050239,0.0028019\right]}$ & $\mathbf{0.041205}$ & $\mathbf{\left[0.01038,0.09576\right]}$ \tabularnewline 
\hline 
\end{tabular} 

:
% \begin{tabular}{cc|cc|cc} 
\hline 
 \multirow{2}{*}{Solver} & \multirow{2}{*}{Model} & \multicolumn{2}{c|}{Support RMSE}  & \multicolumn{2}{c}{Challenge RMSE} \tabularnewline & & Median & Confidence Interval & Median & Confidence Interval \tabularnewline\hline 
\hline 
\multirow{3}{*}{WalkSAT/SKC} & Exp. & $0.0013414$ & $\left[0.00066148,0.0023947\right]$ & $0.76758$ & $\left[0.34781,1.633\right]$ \tabularnewline 
 & RootExp. & $0.0010909$ & $\left[0.00044449,0.0025122\right]$ & $0.16244$ & $\left[0.026794,0.4174\right]$ \tabularnewline 
 & Poly. & $\mathbf{0.001276}$ & $\mathbf{\left[0.00050239,0.0028019\right]}$ & $\mathbf{0.041205}$ & $\mathbf{\left[0.01038,0.09576\right]}$ \tabularnewline 
\hline 
\end{tabular} 



\par\end{centering}
\caption{\label{tab:Bootstrap-model-RMSE} \bestBoot{Median and 95\% bootstrap intervals
of model prediction RMSEs for the medians of the \randomizedAlgorithm{per-instance median} running times.
\medianInterval{To calculate the median RMSEs we used the geometric mean of the intervals for the medians of
the \randomizedAlgorithm{per-instance median} running times for each instance size. However, the bootstrap
confidence intervals directly capture both sources of uncertainty by reporting the 2.5 and 97.5
quantiles of the lower and upper bounds on the RMSEs, respectively.} The models yielding the most
accurate predictions (as per median challenge RMSE) are shown in boldface. }}
\end{table*}


\begin{table*}[tb]
\noindent \begin{centering}
\begin{tabular}{ccccc}
\hline 
\multirow{2}{*}{Solver} & \multirow{2}{*}{$n$} & Predicted confidence intervals & \multicolumn{2}{c}{Observed median run-time}\tabularnewline
 &  & Exp. model  & Point estimates  & Confidence intervals\tabularnewline
\hline 
\hline 
\multirow{7}{*}{WalkSAT/SKC} & 200 & $\left[0.002754,0.004125\right]$ & $0.002093$ & $\left[0.001899,0.002539\right]$ \tabularnewline 
 & 250 & $\mathbf{\left[0.004279,0.005926\right]}$ & $0.004457$ & $\left[0.003664,0.005377\right]$ \tabularnewline 
 & 300 & $\mathbf{\left[0.006648,0.008532\right]}$\textbf{*} & $0.007494$ & $\left[0.006795,0.008479\right]$ \tabularnewline 
 & 350 & $\mathbf{\left[0.01034,0.01227\right]}$ & $0.01094$ & $\left[0.009792,0.01258\right]$ \tabularnewline 
 & 400 & $\mathbf{\left[0.01563,0.01822\right]}$ & $0.01825$ & $\left[0.01601,0.0201\right]$ \tabularnewline 
 & 450 & $\mathbf{\left[0.0228,0.02791\right]}$ & $0.02414$ & $\left[0.02149,0.02995\right]$ \tabularnewline 
 & 500 & $\mathbf{\left[0.03263,0.04218\right]}$ & $0.03651$ & $\left[0.03177,0.04233\right]$ \tabularnewline 
\hline 
\end{tabular} 

\begin{tabular}{ccccc}
\hline 
\multirow{2}{*}{Solver} & \multirow{2}{*}{$n$} & Predicted confidence intervals & \multicolumn{2}{c}{Observed median run-time}\tabularnewline
 &  & RootExp. model  & Point estimates  & Confidence intervals\tabularnewline
\hline 
\hline 
\multirow{7}{*}{WalkSAT/SKC} & 200 & $\mathbf{\left[0.002052,0.00334\right]}$ & $0.002093$ & $\left[0.001899,0.002539\right]$ \tabularnewline 
 & 250 & $\mathbf{\left[0.003717,0.005335\right]}$ & $0.004457$ & $\left[0.003664,0.005377\right]$ \tabularnewline 
 & 300 & $\mathbf{\left[0.006362,0.008301\right]}$ & $0.007494$ & $\left[0.006795,0.008479\right]$ \tabularnewline 
 & 350 & $\mathbf{\left[0.01049,0.01242\right]}$ & $0.01094$ & $\left[0.009792,0.01258\right]$ \tabularnewline 
 & 400 & $\mathbf{\left[0.01612,0.01876\right]}$ & $0.01825$ & $\left[0.01601,0.0201\right]$ \tabularnewline 
 & 450 & $\mathbf{\left[0.02324,0.02848\right]}$ & $0.02414$ & $\left[0.02149,0.02995\right]$ \tabularnewline 
 & 500 & $\mathbf{\left[0.03229,0.0418\right]}$ & $0.03651$ & $\left[0.03177,0.04233\right]$ \tabularnewline 
\hline 
\end{tabular} 

\begin{tabular}{ccccc}
\hline 
\multirow{2}{*}{Solver} & \multirow{2}{*}{$n$} & Predicted confidence intervals & \multicolumn{2}{c}{Observed median run-time}\tabularnewline
 &  & Poly. model  & Point estimates  & Confidence intervals\tabularnewline
\hline 
\hline 
\multirow{7}{*}{WalkSAT/SKC} & 200 & $\mathbf{\left[0.001369,0.002609\right]}$\textbf{*} & $0.002093$ & $\left[0.001899,0.002539\right]$ \tabularnewline 
 & 250 & $\mathbf{\left[0.003143,0.004819\right]}$ & $0.004457$ & $\left[0.003664,0.005377\right]$ \tabularnewline 
 & 300 & $\mathbf{\left[0.006063,0.00811\right]}$ & $0.007494$ & $\left[0.006795,0.008479\right]$ \tabularnewline 
 & 350 & $\mathbf{\left[0.01061,0.01266\right]}$ & $0.01094$ & $\left[0.009792,0.01258\right]$ \tabularnewline 
 & 400 & $\mathbf{\left[0.01661,0.01927\right]}$ & $0.01825$ & $\left[0.01601,0.0201\right]$ \tabularnewline 
 & 450 & $\mathbf{\left[0.02363,0.02901\right]}$ & $0.02414$ & $\left[0.02149,0.02995\right]$ \tabularnewline 
 & 500 & $\mathbf{\left[0.03187,0.04135\right]}$ & $0.03651$ & $\left[0.03177,0.04233\right]$ \tabularnewline 
\hline 
\end{tabular} 



% \input{table_Bootstrap-intervals}
\par\end{centering}

\caption{\label{tab:Bootstrap-intervals-support} 95\% bootstrap confidence intervals
for the medians of the \randomizedAlgorithm{per-instance median} running time predictions and observed running times on random 3-SAT instances at the phase transition.
The instance sizes shown here are those used for fitting the models.
Bootstrap intervals on predictions that are weakly consistent
with the observed point estimates are shown in boldface
%\medianInterval{those that are consistent are marked by plus signs ({+}),}
and those that are strongly consistent are marked
by asterisks ({*}).}
%and those that fully contain the confidence intervals on
%observations are marked by asterisks ({*}).}
\end{table*}

\begin{table*}[tb]
\noindent \begin{centering}
\begin{tabular}{ccccc}
\hline 
\multirow{2}{*}{Solver} & \multirow{2}{*}{$n$} & Predicted confidence intervals & \multicolumn{2}{c}{Observed median run-time}\tabularnewline
 &  & EXP. model  & Point estimates  & Confidence intervals\tabularnewline
\hline 
\hline 
\multirow{2}{*}{p15S363} & 2500 & $\mathbf{\left[50.14,154\right]}$ & $41.9$ & $\left[35.35,64.72\right]$ \tabularnewline 
 & 3000 & $\mathbf{\left[102.2,525.1\right]}$\textbf{*} & $135$ & $\left[107.2,177.2\right]$ \tabularnewline 
\hline 
\end{tabular} 

\begin{tabular}{ccccc}
\hline 
\multirow{2}{*}{Solver} & \multirow{2}{*}{$n$} & Predicted confidence intervals & \multicolumn{2}{c}{Observed median run-time}\tabularnewline
 &  & Poly. model  & Point estimates  & Confidence intervals\tabularnewline
\hline 
\hline 
\multirow{2}{*}{p15S363} & 2500 & $\mathbf{\left[38.07,54.13\right]}$\textbf{\#} & $41.9$ & $\left[35.35,64.72\right]$ \tabularnewline 
 & 3000 & $\left[55.77,89.02\right]$ & $135$ & $\left[107.2,177.2\right]$ \tabularnewline 
\hline 
\end{tabular} 

\begin{tabular}{ccccc}
\hline 
\multirow{2}{*}{Solver} & \multirow{2}{*}{$n$} & Predicted confidence intervals & \multicolumn{2}{c}{Observed median run-time}\tabularnewline
 &  & SQRTEXP. model  & Point estimates  & Confidence intervals\tabularnewline
\hline 
\hline 
\multirow{2}{*}{p15S363} & 2500 & $\mathbf{\left[42.94,61.24\right]}$ & $41.9$ & $\left[35.35,64.72\right]$ \tabularnewline 
 & 3000 & $\mathbf{\left[72.4,118.1\right]}$ & $135$ & $\left[107.2,177.2\right]$ \tabularnewline 
\hline 
\end{tabular} 



% \input{table_Bootstrap-intervals}
\par\end{centering}

\caption{\label{tab:Bootstrap-intervals-challenge} 95\% bootstrap confidence intervals
for the medians of the \randomizedAlgorithm{per-instance median} running time predictions and observed running times on random 3-SAT instances at the phase transition.
The instance sizes shown here are larger than those used for fitting the models.
Bootstrap intervals on predictions that are weakly consistent
with the observed data are shown in boldface
%\medianInterval{those that are consistent are marked by plus signs ({+}),}
and those that are strongly consistent are marked
by asterisks ({*}).}
%and those that fully contain the confidence intervals on
%observations are marked by asterisks ({*}).}
\end{table*}


\section{Conclusion}

In this report, we presented an empirical analysis of the scaling
behaviour of WalkSAT/SKC on random 3-SAT instances at the phase transition. We found
the Exp model over-estimates the data, the RootExp model tends to fit the data, and the Poly model fits the data very well.

\bibliographystyle{plain}
\begin{thebibliography}{1}

\bibitem{dubois2015on}
J{\'e}r{\'e}mie Dubois-Lacoste, Holger~H. Hoos, and Thomas St{\"u}tzle.
\newblock On the empirical scaling behaviour of state-of-the-art local search
algorithms for the {E}uclidean {TSP}.
\newblock In {\em Proceedings of the 17th Genetic and Evolutionary Computation Conference}, (GECCO '15), pages 377--384, 2015.

\bibitem{hoos2009bootstrap}
Holger~H. Hoos.
\newblock A bootstrap approach to analysing the scaling of empirical run-time
data with problem size.
\newblock Technical report, Technical Report TR-2009-16, Department of Computer Science, University of British
Columbia, 2009.

\bibitem{hoos2014empirical}
Holger~H. Hoos and Thomas St{\"u}tzle.
\newblock On the empirical scaling of run-time for finding optimal solutions to
the travelling salesman problem.
\newblock {\em European Journal of Operational Research}, 238(1):87--94, 2014.

\end{thebibliography}

\end{document}
